\documentclass{article}
\usepackage[utf8]{inputenc}
\usepackage[english]{babel}
 
\usepackage{multicol}
\special{papersize=8.5in,11in}
\usepackage[margin=0.5in]{geometry}
\usepackage{mathptmx}
\usepackage{lipsum}

\begin{document}

\noindent\LARGE{\textbf{On the effect of rotor wake turbulence on bat lungs and fish swim bladders}}
\vspace{0.08cm}
\normalsize

\hspace{0.5cm}Dorien Villafranco

\small\hspace{0.5cm}\textit{Department of Mechanical Engineering, Boston University, 110 Cummington Mall,}

\hspace{0.5cm}\textit{Boston, Massachusetts 02215}
\normalsize
\vspace{0.15cm}

\hspace{0.5cm}Jonathan Russell
\small

\hspace{0.5cm}\textit{Department of Mechanical Engineering, Boston University, 110 Cummington Mall,}

\hspace{0.5cm}\textit{Boston, Massachusetts 02215}


\vspace{5cm}
\begin{multicols}{2}
\noindent\textbf{I. INTRODUCTION}

This paper presents the effects of rotor wake turbulence on two animals, bats and fish, which are in constant contact with such turbulent distortions. It is known that bat mortality is increased near moving turbine blades usually found in wind farms (WFs). xx  The literature suggests two leading hypotheses for the mortality of bats in the vicinity of wind farms. It is supposed that the bats are either killed by direct contact with the turbine blades or by barotrauma. xx Barotrauma is an occurrence in which a sudden change in the surrounding air-pressure causes tissue damage to biological structures which contain air in the bat's body such as the lungs. A recent study has reported that barotrauma may be the cause for about 90$\%$ of bat deaths. The study demonstrates that the aforementioned proportion of bats were all found to have lesions associated with barotrauma. xx
\lipsum[1-2]

\end{multicols}
\end{document}